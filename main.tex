\documentclass[%
%reprint,
%superscriptaddress,
%groupedaddress,
%unsortedaddress,
%runinaddress,
%frontmatterverbose, 
% preprint,
%preprintnumbers,
 nofootinbib,
%nobibnotes,
%bibnotes,
 amsmath,amssymb,
 aps,
%pra,
%prb,
%rmp,
%prstab,
%prstper,
%floatfix,
 twocolumn
]{revtex4-2}


\usepackage{aas_macros}
\usepackage{amssymb}
\usepackage{amsmath}
\usepackage{array}
\usepackage{color,units}
\usepackage[dvipsnames]{xcolor} % for more colours!
\usepackage{lineno}
\usepackage{dcolumn}
\usepackage{longtable}
\usepackage[normalem]{ulem} %% for striking out text
\usepackage{subfigure}
\usepackage[T1]{fontenc}
\usepackage[breaklinks]{hyperref}
\usepackage{booktabs}
\usepackage{xspace} 
\usepackage{bm}% bold math
\usepackage{graphicx} % Include figure files
% \usepackage{authblk}

\graphicspath{{images/}} %Setting the graphicspath

% symbols
\DeclareSymbolFont{extraup}{U}{zavm}{m}{n}
\DeclareMathSymbol{\varheart}{\mathalpha}{extraup}{86}
\DeclareMathSymbol{\vardiamond}{\mathalpha}{extraup}{87}


% Keywords
\newcommand{\bilby}{{\sc \href{https://lscsoft.docs.ligo.org/bilby/}{\texttt{Bilby}}}\xspace}
\newcommand{\bilbypipe}{{\sc bilby\_pipe}\xspace}
\newcommand{\pbilby}{{\sc pBilby}\xspace}
\newcommand{\dynesty}{{\sc dynesty}\xspace}
\newcommand{\cpnest}{{\sc cpnest}\xspace}
\newcommand{\ptemcee}{{\sc ptemcee}\xspace}
\newcommand{\gwpy}{{\sc \href{https://gwpy.github.io/}{\texttt{GWpy}}}\xspace}
\newcommand{\imrphenomp}{{\sc IMRPhenomPv2}\xspace}
\newcommand{\seob}{{\sc SEOBNRv4PHM}\xspace}
\newcommand{\gstlal}{{\sc GstLAL}\xspace}
\newcommand{\pycbc}{{\sc \href{https://pycbc.org/}{\texttt{PyCBC}}}\xspace}


% math keywords 
\newcommand{\fancytext}[1]{{\relax\ifmmode#1\else $#1$\fi}\xspace}
\newcommand{\mathcmd}[1]{{\sc \relax\ifmmode#1\else $#1$\fi}\xspace}
\newcommand{\bcr}{\mathcmd{\rho_\text{BCR}}}
\newcommand{\pycbcstat}{\mathcmd{\rho_\text{PC}}}
\newcommand{\snr}{\mathcmd{\rho}}
\newcommand{\psd}{\mathcmd{P(f)}}
\newcommand{\msun}{\mathcmd{\text{M}_\odot}}
\newcommand{\parameters}{\mathcmd{\vec{\theta}}}
\newcommand{\prior}{\mathcmd{\pi(\parameters)}}
\newcommand{\template}{\mathcmd{\mu(\parameters)}}
\newcommand{\fap}{\mathcmd{\text{FAP}}}

\newcommand{\pastro}{\fancytext{p_\text{astro}}}
\newcommand{\pastrobcr}{\fancytext{p_\text{astro}^{\text{BCR}}}}
\newcommand{\untunedpastrobcr}{\fancytext{p_\text{astro}^{\text{BCR}\prime}}}
\newcommand{\pastroext}{\fancytext{p_\text{astro}^{\text{ext}}}}
\newcommand{\pval}{\fancytext{\text{p-value}}}
% Table settings
\renewcommand{\aboverulesep}{0pt}
\renewcommand{\belowrulesep}{0pt}
% \setlength\cellspacetoplimit{2pt}
% \setlength\cellspacebottomlimit{2pt}

\newcolumntype{C}{>{\centering\arraybackslash}X}
\newcolumntype{L}{>{\arraybackslash}X}

% Aesthetic colours
\definecolor{dodgerblue}{HTML}{1E90FF}
\definecolor{viennared}{HTML}{DA0A14}
\definecolor{ctorange}{HTML}{FF6C0C}
\definecolor{wales}{HTML}{ff0038}
\definecolor{benettongreen}{HTML}{009421}
\definecolor{valenciacfred}{HTML}{ee3524}
\definecolor{barcelonafcgold}{HTML}{edbb00}
\definecolor{jam}{HTML}{A50B5E}
\definecolor{austriawien}{HTML}{441678}
\definecolor{italia90green}{HTML}{009966}
\definecolor{ferrarired}{HTML}{ff2800}
\definecolor{gray}{HTML}{F0F0F0}
\definecolor{LightCyan}{rgb}{0.88,1,1}
\newcolumntype{a}{>{\columncolor{gray}}c}
\newcolumntype{b}{>{\columncolor{white}}c}

% author comments
\newcommand{\avi}[1]{\textcolor{orange}{[AV: #1]}}
\newcommand{\rs}[1]{\textcolor{red}{[RS: #1]}}
\newcommand{\et}[1]{\textcolor{blue}{[ET: #1]}}


% %%% AFFILIATIONS
% %------ affiliation shortcuts
% \newcommand{\SPAno}{1}
% \newcommand{\OzGravMonashno}{2}
% \newcommand{\CITno}{3}
% \newcommand{\MITno}{4}
% \newcommand{\Kavlino}{5}

% %--------------------------
% \newcommand{\SPA}{School of Physics and Astronomy, Monash University, Clayton VIC 3800, Australia}
% \newcommand{\OzGravMonash}{OzGrav: The ARC Centre of Excellence for Gravitational Wave Discovery, Clayton VIC 3800, Australia}
% \newcommand{\MIT}{LIGO Laboratory, Massachusetts Institute of Technology, Cambridge, MA 02139, USA}
% \newcommand{\Kavli}{Department of Physics and Kavli Institute for Astrophysics and Space Research, Massachusetts Institute of Technology, \\ 77 Massachusetts Ave, Cambridge, MA 02139, USA}
% \newcommand{\CIT}{LIGO Laboratory, California Institute of Technology, Pasadena, CA 91125, USA}
% %%%

\begin{document}


\title{A Bayesian ranking statistic to find high-mass black holes in LIGO--Virgo data}


% 

%------ affiliation shortcuts
\newcommand{\SPAno}{School of Physics and Astronomy, Monash University, Clayton VIC 3800, Australia}
\newcommand{\OzGravMonashno}{OzGrav: The ARC Centre of Excellence for Gravitational Wave Discovery, Clayton VIC 3800, Australia}
\newcommand{\CITno}{LIGO Laboratory, California Institute of Technology, Pasadena, CA 91125, USA}
\newcommand{\MITno}{LIGO Laboratory, Massachusetts Institute of Technology, Cambridge, MA 02139, USA}
\newcommand{\Kavlino}{Department of Physics and Kavli Institute for Astrophysics and Space Research, Massachusetts Institute of Technology, 77 Massachusetts Ave, Cambridge, MA 02139, USA}


\author{Avi~Vajpeyi}
\affiliation{\SPAno}
\affiliation{\OzGravMonashno}
\author{Rory~Smith}
\affiliation{\SPAno}
\affiliation{\OzGravMonashno}
\author{Eric~Thrane}
\affiliation{\SPAno}
\affiliation{\OzGravMonashno}
\author{Gregory~Ashton}
\affiliation{\SPAno}
\affiliation{\OzGravMonashno}
\author{Thomas~Alford}
\affiliation{\CITno}
\author{Sierra~Garza}
\affiliation{\CITno}
\author{Maximiliano~Isi}
\affiliation{\MITno}
\affiliation{\Kavlino}
\author{Jonah~Kanner}
\affiliation{\CITno}
\author{T. J.~Massinger}
\affiliation{\CITno}
\author{Liting~Xiao}
\affiliation{\CITno}

% \affiliation[\OzGravMonashno]{}
% \affiliation[\MITno]}
% \affiliation[\Kavlino]{}
% \affiliation[\CITno]{}

%\author[1]{Test}
%\affil[1]{Test Affil}


\date{\today}

\begin{abstract}
The detection of intermediate-mass black holes ( $10^2-10^6$~\msun) will shed light on the formation of supermassive black holes and thus galaxy formation. Although LIGO is sensitive to the merger of binary black holes with total masses up to $400$~\msun, only 4 of their 50 detections have a total mass $>100$~\msun with >95\% credibility. A possible explanation for the absence of intermediate-mass events may be their misclassification as short-duration instrumental noise transients. Short-duration instrumental transients mimic the short-duration gravitational-wave signals from intermediate-mass binary black hole mergers. 
\et{I think the preceding text is misleading. We make it sound like we expect a uniform distribution of total mass, and so it's a surprising that we don't see IMBH. However, in reality, it is theoretically challenging to make IMBH. You need an exceptionally massive progenitor star to avoid pair instability supernova. Or, maybe you can make IMBH from hierarchical mergers or accretion. No one expects them to be common, but they are  interesting because of their connection to supermassive black holes. I suggest rewriting the preceding text to explain how IMBH might form and why they are interesting to look for.}
Here we demonstrate a ranking statistic utilising Bayesian inference for the detection of high-mass binary black hole mergers (with a total mass $>55$~\msun). \et{Claims of beating matched filtering not yet borne out.} We apply this technique on the high-mass triggers during LIGO's second observing run to search for previously unresolved gravitational-wave signals from high-mass binary black holes. Our analysis does not discover new gravitational-wave events. However, we find support for some previously identified borderline detections.
\et{Don't oversell Bayesian measure since our method is only semi-Bayesian.}
\end{abstract}

\maketitle


%%%%%%---SECTIONS---%%%%%%%%%%%%
\section{Introduction}
Since the 1970s, there has been a steady accumulation of evidence for stellar mass ($\text{M}_\text{BH} < 10^{2} \ \msun$) and supermassive black holes ($\text{M}_\text{BH} > 10^{5} \ \msun$)~\cite{Webster:1972:Natur, Balick:1974:ApJ, Ghez:1998:ApJ, Genzel:2010:RvMP, Abbott:2019:PhRvX, EventHorizonTelescopeCollaboration:2019:ApJL, Abbott:2020:arXiv}. However, there is a deficiency of observational evidence for black holes in the intermediate-mass range $10^{2} - 10^{5}\ \text{M}_{\odot}$. Astronomers have detected a handful of intermediate-masses candidates using reverberation mapping, taking direct kinematic measurements, applying  M-$\sigma$ and M-L relations, and also studying  X-ray luminosity and spectra~\cite{Greene:2004:ApJ, Graham:2013:ApJ, Mezcua:2017:IJMPD, Koliopanos:2017:mbhe, Lin:2020:ApJL, Greene:2020:ARA&A}. However, these observations are highly uncertain and can be attributed to sources without intermediate-mass black holes~\cite{kiziltan2017, freire2017}. In the last few years, astronomers have found definitive evidence for two intermediate-mass black holes, (i) the $142^{+28}_{-16}\ \msun$ remnant observed from the gravitational wave event GW190521~\cite{Abbott:2020:PhRvL}, and the $5.5^{+1.7}_{-0.9}\times10^{4}\ \msun$ black hole identified from gravitational lensing in the light curve of GRB950830~\cite{paynter_evidence_2021}. The concrete discovery of more intermediate-mass black holes will bridge the observational gap and illuminate our understanding of galaxy and supermassive black hole formation. Additionally, a catalog of observed intermediate-mass black holes can act as probes to their formation environments, such as in the accretion disks of active galactic nuclei~\cite{}, the centers of dense stellar clusters~\cite{}, and even Population-III stars~\cite{}. 


% Given the existence of supermassive black holes detected in the early universe, it is likely for intermediate-mass black holes to have also existed~\cite{Banados:2018:Natur}. However, we only have a handful of promising candidates~\cite{Greene:2004:ApJ, Graham:2013:ApJ, Mezcua:2017:IJMPD, Koliopanos:2017:mbhe, Lin:2020:ApJL, Greene:2020:ARA&A}, and two confirmed intermediate-mass black holes, (i) the $142^{+28}_{-16}\ \msun$ remnant observed from the gravitational wave event GW190521~\cite{Abbott:2020:PhRvL}, and the $5.5^{+1.7}_{-0.9}\times10^{4}\ \msun$ black hole identified from gravitational lensing in the light curve of GRB950830~\cite{paynter_evidence_2021}. The concrete discovery of more intermediate-mass black holes will bridge the observational gap and illuminate our understanding of galaxy and supermassive black hole formation. Additionally, a catalog of observed intermediate-mass black holes can act as probes to their formation environments, such as in the accretion disks of active galactic nuclei~\cite{}, the centers of dense stellar clusters~\cite{}, and even Population-III stars~\cite{}. 

% However, measuring the mass of black holes can be challenging. Several observational techniques to identify intermediate mass black holes

% It is possible to measure the mass of nearby supermassive black holes by directly observing the dynamic motion of stars nearby the supermassive black hole (such as Sgr A$^*$~). However, this dynamic measurement method is challenging to utilize for  intermediate-mass black holes as their sphere of influence is too small to resolve motions~\cite{}. Additionally, dynamic-measurements that have alluded to  intermediate-mass candidates, e.g. the candidate at the center of the globular cluster 47 Tucanae~\cite{kiziltan2017}, can also be attributed to simpler models without a central intermediate-mass black hole~\cite{freire 2017}. }}

% \rs{\sout{In a similar vein, although the M-$\sigma$ and M-L relations (relations between the spheroidal bulge of a galaxy's velocity dispersion $\sigma$,  luminosity L, and its central black hole's mass M) can estimate the mass of supermassive black holes~\cite{Ferrarese & Merritt 2000 and Gebhardt et al. 2000}, they cannot accurately estimate the intermediate-masses. The M-$\sigma$ and M-L relations are not calibrated for low-mass galaxies that might host an central intermediate-mass black hole~\cite{J. E. Greene and L. C. Ho The Mass Function of Active Black Holes in the Local Universe, E. C. Moran, K. Shahinyan, H. R. Sugarman, D. O. Vélez and M. Eracleous Black Holes At the Centers of Nearby Dwarf Galaxies}. Table 2 of~\citet{mercuza} displays a list of intermediate-mass candidates found with these relations. }}

% \rs{\sout{Another method to estimate large black hole masses is with reverberation mapping, which estimates black hole virial masses by studying their nearby gasses' velocities~\cite{reverberation mapping}. Reverberation mapping has shown to estimate both supermassive black hole and intermediate-mass black hole virial masses successfully. However, this method fails to measure the systematic uncertainties on the lower-end of the intermediate-mass spectrum~\cite{}. Additionally, cloud-variability and the possibility of anisotropic emission deride the credibility of intermediate-mass candidates found with reverberation mapping~\cite{}. }}

% The most promising intermediate-mass candidates found with electromagnetic-spectra are Ultra-Luminous X-ray sources with luminosities $> 10^{39} \text{erg s}^{-1}$ (e.g. HLX-X1)~\cite{}. \rs{\sout{While stellar mass black holes are too small to generate such luminosities with Eddington accretion, intermediate mass black holes can accrete at rates large enough to account for the high luminosities~\cite{}. However, some argue that some stellar mass black holes may exceed the Eddington limit~\cite{}, and others argue that the X-ray emission might be from a collimated jet leading to an inaccurate mass calculation~\cite{}.}}

Gravitational waves from compact binaries coalesces (CBCs) are clean channels to measure BH masses. Mergers involving either a binary system with at least one intermediate-mass component, or a system whose merger results in an intermediate-mass remnant will have large initial total masses $\text{M}_{T}$. As a binary's $\text{M}_{T}$ is associated with its merger frequency, $f\sim \text{M}_{T}^{-1}$, systems with large $\text{M}_{T}$ have very low merger frequencies $f < 100\ \text{Hz}$. Hence, ground based gravitational wave detectors ($\sim 10 - 1000\ \text{Hz}$) are sensitive to the last milliseconds of merging systems with $100 \msun < \text{M}_{T} < 400 \msun$~\cite{}, while space-based detectors ($\sim 0.1 - 10\ \text{Hz}$) can study the full signals of merging systems with $10^4 \msun < \text{M}_{T} < 10^7 \msun$~\cite{}. 

Because of high-total mass systems' short-durations in ground-based gravitational wave data steams, handling data quality is critical to their detection. The low-frequency ranges of ground-based observatories are plagued with incoherent non-stationary terrestrial artefacts, called glitches~\cite{ pycbc_short_duration_transients, pe_with_glitch, blip_glitches}. Some glitches, similar to signals from high total mass mergers, last for a fraction of a second, making them difficult to distinguish from the signals. These glitches that mimic astrophysical signals can severely decrease the confidence in detection of true gravitational-waves from high total mass mergers. 

Although a significant fraction of the glitches can be removed by testing them for coherence amongst various ground based detectors and performing matched-filtering, these methods are insufficient to remove all the glitches. One method to account for more glitches while searching for high total mass CBC gravitational-waves is by utilizing the astrophysical Bayesian odds~\cite{bci, BCR1, BCR2, bcr_gw151216, bayesian_odds}. A true Bayesian odds, calculated without using bootstrap techniques, can provide more accurate estimate of significance that does note depend on the search pipeline~\cite{BCR2, bcr_gw151216,  bayesian_odds}. In this paper, we utilize a Bayesian method, called the Bayesian Coherence Ratio \bcr~\cite{BCR1}, to rank the candidate gravitational-wave signals from high-mass compact binary coalescences (systems with total masses in the range of $55-500\ \msun$) in the detector data recorded during O2. Although the \bcr, utilizing bootstrap techniques, does not provide the true Bayesian odds,  it utilizes Bayesian evidences which describe the explicit probability of data under the hypothesis that it contains coherent signals vs incoherent glitches.


\avi{Need to discuss IAS, PyCBC etc here as well}

We find that (a) high-mass events reported in the GWTC-1, including GW170729 (the least significant event in GWTC-1) have high significance; (b) high-mass events detected from the IAS group have differing levels of significance; \et{differing from what? edit this passage to state more clearly that we find statistical support for some of their candidates, but not for others} and that (c) our ranking statistic does not identify any intermediate-mass black holes, but does identify an unreported stellar-mass binary black hole candidate, 170222 \et{State statistical significance.}.

The remainder of this paper is structured as follows. We outline our methods, including details of our ranking statistic and the retrieval of our candidate events in Section~\ref{sec:method}. We present details on the implementation of our analysis in Section~\ref{sec:Analysis}. Finally, we present our results in Section~\ref{sec:Results}, and discuss these results in the context of the significance of gravitational-wave candidates in Section~\ref{sec:Conclusion}.



\section{Method\label{sec:method}}

The standard framework to identify CBC gravitational-wave signals hidden in data is by quantifying the significance of candidates with null-hypothesis significance testing. In this framework, the candidates' ranking statistic is compared against a background distribution in a frequentist approach. \rs{add a sentence describing what the ranking stats are for the different pipelines (key words: matched filtering, SNR etc...) and add citations!} On the other hand, the standard framework for performing parameter estimation and model selection in gravitational-wave astronomy is Bayesian inference. This work utilizes Bayesian inference to calculate the Bayesian Coherence odds-ratio~\cite{BCR1}, \bcr, of high-mass candidates in LIGO's second observing run. We use the \bcr not as an odds-ratio but instead as a ranking statistic, a step toward building a unified Bayesian framework to search for candidates and estimate their parameters.

\subsection{The Bayesian Coherence Ratio}

Bayes theorem states that the posterior probability distribution $p(\parameters|d,\mathcal{H})$ for data $d$ and a vector of parameters \parameters that describe a model which quantifies a hypothesis $\mathcal{H}$, is given by
\begin{equation}
p(\parameters|d,\mathcal{H}) = \frac{\mathcal{L}(d|\parameters, \mathcal{H}) \ \pi(\parameters | \mathcal{H})}{\mathcal{Z}(d|\mathcal{H})}\ , 
\end{equation}
where $\mathcal{L}(d|\parameters, \mathcal{H})$ is the likelihood of the data given the parameters \parameters and the hypothesis, $\pi(\parameters | \mathcal{H})$ is the prior probability of the parameters, and finally,
\begin{equation}
    \mathcal{Z}(d|\mathcal{H}) = \int\limits_{\parameters} \ \mathcal{L}(d|\parameters,\mathcal{H}) \ \pi(\parameters | \mathcal{H}) \ d\parameters
\end{equation} is the likelihood after marginalizing over the parameters \parameters.  To compare two hypotheses $\mathcal{H}_A$ and $\mathcal{H}_B$ with the Bayes theorem one can calculate an odds-ratio
\begin{equation}
    \mathcal{O}^A_B = \frac{\mathcal{Z}^A\ \pi(\parameters^A)}{\mathcal{Z}^B\ \pi(\parameters^B)}\ ,
\end{equation}
where $\mathcal{Z}^A$ and $\mathcal{Z}^B$ are the shorthand for the evidences  $\mathcal{Z}(d|\mathcal{H}_A)$ and $\mathcal{Z}(d|\mathcal{H}_B)$. The odds-ratio can tell us which of the two hypotheses is more likely. For example, if $\mathcal{O}^A_B >> 1$, then this odds ratio indicates that the $\mathcal{H}_A$ describes the data much better than $\mathcal{H}_B$. 

The \bcr is a Bayesian odds-ratio like the above, of a coherent signal hypotheses $\mathcal{H}_S$ and an incoherent instrumental feature hypothesis $\mathcal{H}_I$ (the null-hypotheses) for a network of $D$ detectors. $\mathcal{H_I}$ states that each detector $i$ has either pure stationary Gaussian noise $\mathcal{H}_N$ or Gaussian noise and an incoherent noise transient (glitch). Taking $Z^S$, $Z^G_i$ and $Z^N_i$ as the Bayesian evidences (marginalised likelihoods, definitions in Appendix~\ref{apdx} for $\mathcal{H}_S$, $\mathcal{H}_N$, and $\mathcal{H}_G$, the \bcr is given by
\begin{equation}
\label{eq:bcr}
\bcr = \frac{\alpha Z^S}{\prod\limits^D_{i=1} \ [\beta Z^G_i + (1-\beta)Z^N_i]}\ ,
\end{equation}
where $\alpha$ and $\beta$ are the prior-odds of obtaining a signal or a glitch from a stretch of data. The prior-odds can be defined more explicitly as 
\begin{itemize}
    \item $\alpha=P(\mathcal{H}_S)/P(\mathcal{H}_I)$, the prior-odds for obtaining a coherent signal versus an incoherent instrumental feature.
    \item $\beta=P(\mathcal{H}_G| \mathcal{H}_I)$, the prior-odds for obtaining a glitch assuming there is an incoherent instrumental feature.
\end{itemize}

When $\mathcal{H}_S$ and $\mathcal{H}_I$ are precisely described and the correct prior-odds are known, the \bcr is a Bayesian odds-ratio. As an odds-ratio, the \bcr is the optimal discriminator between coherent signals and incoherent instrumental features. However, as the priors-odds are unknown, it is invalid to use the \bcr as an odds-ratio to make an informed decision about whether a candidate is from an astrophysical or terrestrial source. Instead of interpreting the \bcr as a Bayesian odds-ratio, it can be used as a ranking statistic. Using the \bcr as a ranking statistic we can obtain a frequentist significance of a candidate \bcr-value measured against a background \bcr distribution. 

When using the \bcr as a detection statistic, the physical interpretation of the prior-odds is lost. Hence, the prior-odds are empirically tuned to maximise the separation between the \bcr distribution of the background (expected to favour the $\mathcal{H}_I$ hypothesis) and the \bcr distribution of artificially manufactured simulated signals (expected to favour the $\mathcal{H}_S$ hypothesis). Increasing the separation between the two distributions can improve ability of the \bcr to discriminate candidate events as coherent signals or incoherent instrumental features. The tuning process is described in detail in Appendix~\ref{sec:tuning-prior-odds}. 

\subsection{Calculating the Significance of Candidates}
Candidate \bcr-values are either statistically insignificant compared to the background \bcr distribution, implying the candidate is more probable to be an incoherent instrumental feature (the $\mathcal{H}_I$ null-hypothesis), or statistically significant to the background distribution, indicating the possible presence of an astrophysical signal (the $\mathcal{H}_S$ hypothesis). A false alarm probability with trial factors, \fap, for the candidate \bcr-value can quantify the significance. The \fap is the probability that a candidate originating from a non-astrophysical source can be is falsely identified as a signal.

To calculate the \fap, each candidate \bcr is considered a single statistical trail that can occur at a fixed false alarm probability $f$, where $f$ is the probability of observing a background $\bcr'$ greater than or equal to the candidate \bcr,
\begin{equation}
    f = \frac{\text{Count of } \bcr' \leq \bcr}{\text{Count of } \bcr'} \ .
\end{equation}
The false alarm probability with trails \fap that the \bcr measurement occurs at least once for $N$ trials $(N > 0)$, where $N$ is the number of candidate triggers is
\begin{equation}
    \fap = 1 - (1-f)^N \ .
\end{equation}

Finally, the \fap can be used to construct a \pastro, the probability that a signal is of astrophysical origin~\cite{pastro_1,pastro_2,pastro_3}
\begin{equation}
    \pastro = 1 -  \fap \ .
\end{equation}


\subsection{Data for Analysis}
\avi{Probably need a better section label...}
The LIGO Scientific collaboration operates several search pipelines that scan for gravitational-waves from compact binary mergers such as \texttt{GstLAL}, \texttt{MBTA}, \texttt{SPIIR} and \pycbc~\cite{GWTC1}. The output of \pycbc's search is a list of times and their corresponding \pycbc ranking statistic \pycbcstat values. The \pycbcstat ranking-statistic is akin to the matched-filter signal-to-noise ratio \snr. However, unlike \snr, \pycbcstat includes candidate signal's intrinsic and extrinsic properties and other information that feeds into determining if the signal can have astrophysical origins~\cite{pycbc_og6}. Whenever a local maximum of $\pycbcstat > \snr_\text{T}$, where $\snr_\text{T}$ is some predetermined threshold value, the \pycbc search pipeline produces a single-detector \textit{trigger} associated with the detector and time where the apparent signal in the data has its merger~\cite{pycbc_og6}.

When \pycbc observes a trigger between detectors with coincident parameters and a time of arrival difference less than the gravitational-wave travel time between detectors, the trigger is labelled a \textit{candidate event trigger}, a trigger that may be from astrophysical origins~\cite{pycbc_og1}. To test the pipeline's sensitivity \pycbc also conducts searches for \textit{simulated triggers}, artificial triggers manufactured by injecting signals into the detector data. Finally, to quantify the statistical significance of candidate triggers, \pycbc artificially constructs \textit{background triggers} to compare against the candidate events. These background triggers are coherent signal-free events, constructed by applying relative offsets, or time-slides, between the data of different detectors~\cite{pycbc_og6}. The background trigger's \pycbcstat distribution is used to calculate the candidate trigger's significance,  using null-hypothesis significance testing, under the assumption that all candidate event triggers are due to noise.

Our work demonstrates that the \bcr can be used in the same way as \pycbcstat to measure candidate triggers' statistical significance.  The \bcr can be a powerful ranking statistic as it incorporates information of not only all possible binary black hole systems that might have merged to produce the trigger but also the various incoherent glitches that might cause a false-detection. 

\section{Analysis}\label{sec:Analysis}

\subsection{Acquisition of triggers}
Advanced LIGO's second observing run O2 lasted $38$ weeks~\cite{GWOSC}. The software package, \pycbc~\cite{pycbc_code}, was used by LIGO to process the O2 data in 22 time-frames (approximately 2 weeks for one time-frame) and found several gravitational-wave events and numerous gravitational-wave candidates~\cite{pycbc_og0, pycbc_og1, pycbc_og2, pycbc_og3, pycbc_og4, pycbc_og5, pycbc_og6}. Some candidate events were vetoed to be glitches, while others were rejected due to their low significance. The data is divided into these time-frames because the detector's sensitivity does not stay constant throughout the eight-month-long observing period.

In addition to finding candidate events, \pycbc also identified several million background triggers for each time-frame, by searching background data manufactured by time-sliding data within that time-frame. The background triggers help quantify the candidate events' significance for the respective time-frames. Finally, to test the search's sensitivity, \pycbc produced and searched for thousands of simulated signals. 

For our study, we filter the background, simulated and candidate events to include only high-mass events with masses in the ranges of the parameters presented in Table~\ref{tab:parameters}. A plot of the \pycbc triggers from one time-frame, during April 23 - May 8, 2017, is presented in Fig.~\ref{fig:templateBank}. This figure also depicts the gravitational-wave templates used during the search through this time-frame of data. 

\begin{table}[t]

\caption[BBH parameters correspond to duration $<454\ \text{ms}$]{\label{tab:parameters}High-mass parameter space (parameters correspond to signals with durations $<454 \ \text{ms}$). }
\centering
\begin{tabular}{lrr}
\toprule
  & Minimum & Maximum\\
\midrule
Component Mass 1 [\msun] & 31.54 & 491.68\\
Component Mass 2 [\msun] & 1.32 & 121.01\\
Total Mass [\msun] & 56.93 & 496.72\\
Chirp Mass [\msun] & 8.00 & 174.56\\
Mass Ratio & 0.01 & 0.98\\
\end{tabular}
\end{table}



\begin{figure*}[!ht]

{\centering \includegraphics[width=0.75\linewidth]{images/template_bank.png}

}
% time for chunk 14
% 1176955218, Apr 23, 2017, 4:00 UTC
% 1178294418, May 08, 2017, 16:00 UTC
\caption[High-mass BCR search space.]{The template bank (pink) used by \pycbc to search a section of O2 data from $\text{April 23 - May 8, 2017}$. Our search is constrained to the high-mass parameter space enclosed by the dashed line. The candidate, background and simulated triggers detected in this region of the parameter space during this period are plotted in orange, grey and blue respectively.  \avi{Should I remove the `template bank' and only display the various triggers?}}\label{fig:templateBank}
\end{figure*}


\subsection{Calculating the BCR for triggers}
\avi{Sat 13th: Just realised that some parts of this dont make sense without the material present in Apdx~\ref{sec:bayesianEvidEval}}...
To evaluate $Z^S$, $Z^G_i$ and $Z^N_i$ and calculate the \bcr Eq.~\ref{eq:bcr} for triggers, we carry out Bayesian inference with \bilby~\cite{bilby, bilby_pipe}, employing \dynesty~\cite{dynesty} as our nested sampler. Nested sampling, an algorithm introduced by~\citet{skilling2004, skilling2006}, provides an estimate of the true Bayesian evidence and is often utilized for parameter estimation within the LIGO collaboration~\cite{bilby, bilby_paper, pbilby_paper}.

The most computationally intensive step during Bayesian inference is evaluating the likelihood $\mathcal{L}(d_i|\template)$. To accelerate our analysis, we use a likelihood that explicitly marginalizes over coalescence time, phase at coalescence, and luminosity distance (Eq.~80 from~\citet{intro_to_gw_bayes}). While this marginalized likelihood reduces the run time without introducing errors to our evidence evaluation, it does not generates samples for the marginalized parameters. However, these parameter samples can be calculated as a post-processing step~\cite{intro_to_gw_bayes}.

We set the priors $\pi(\parameters|\mathcal{H}_S)$ and $\pi(\parameters|\mathcal{H}_G)$ to be identical. These priors restrict signals with mass parameters in the ranges presented in Table~\ref{tab:parameters}. The spins are aligned over a uniform range for the dimensionless spin magnitude from $\left[0,1\right]$. The luminosity distance prior assigns probability uniformly in comoving volume, with an upper cutoff of $5\ \text{Gpc}$. The full list of the priors, along with their shapes, limits and boundary conditions are documented in Table~\ref{tab:priors}. 

\begin{table}
    \centering
    \caption{
    Prior settings for the parameters used during our parameter estimation. The definitions of the parameters are documented in \citet{bilby_gwtc}~Table~E1.\label{tab:priors}}
    \begin{tabular}{c c c c}
    \hline
    Parameter & Shape & Limits \\
    \hline
          $\mathcal{M}/\msun$           & Uniform & 7--180  \\
          $q$                           & Uniform & 0.1--1  \\
          $M/\msun$                     & Constraint & 50--500  \\
          $d_\mathrm{L}/\mathrm{Mpc}$   & Comoving & 100--5000  \\
          $a_1$, $a_2$                  & Uniform & 0--1  \\
          $\theta_{JN}$                 & Sinusoidal & 0--$\pi$  \\
          $\psi$                        & Uniform & 0--$\pi$  \\
          $\phi$                        & Uniform & 0--$2\pi$  \\
          ra                            & Uniform & 0--$2\pi$  \\
          dec                           & Cosine & 0--$2\pi$  \\
    \hline
    \end{tabular}
\end{table}

The waveform template we utilize is \imrphenomp, a phenomenological waveform template constructed in the frequency domain that models the in-spiral, merger, and ring-down (IMR) of a compact binary coalescence~\citep{khan2016frequency}. Although there exist gravitational-wave templates such as \seob~\cite{seobnrv4phm} which incorporate more physics, such as information on higher-order modes, we use \imrphenomp as it is computationally inexpensive compared to others.

We take 31 neighboring, off-source, non-overlapping,  4-second  segments of time-series data before the analysis data segment $d_i$ to generate the PSD. We use off-source to avoid the inclusion of a signal in the PSD calculation. A Tukey window with 0.2-second roll-offs is applied to each data segment to suppress spectral leakage after which the segments are fast-Fourier transformed and median-averaged to create a PSD~\cite{ligo_psd}. Like other PSD estimation methods, this method adds statistical uncertainties to the PSD~\cite{psd_student_t, chatziioannou2019noise}. To marginalize over the statistical uncertainty, we use the median-likelihood presented by~\citet{psd_student_t} as a post-processing step. 
\avi{Greg would like me to discuss the effect of the PSD marginalization in more detail}


We take 31 neighboring off-source non-overlapping  4-second  segments of time-series data before the analysis data segment $d_i$ to generate the PSD.  A Tukey window with a 0.2-second roll-off is applied to each data segment to suppress spectral leakage. After this  the segments are fast-Fourier transformed and median-averaged to create a PSD~\cite{ligo_psd}. Like other PSD estimation methods, this method adds statistical uncertainties to the PSD~\cite{psd_student_t, chatziioannou2019noise, Biscoveanu:2020:PhRvD}. To marginalize over the statistical uncertainty, we use the median-likelihood presented by~\citet{psd_student_t} as a post-processing step. We find that this post-processing step improves the search efficiency by $x\%$ the details of this calculation are presented in the Appendix~\ref{sec:psd-marginalization}.

Finally, we neglect detector calibration uncertainty and acquire data from from the gravitational-wave Open Science Center~\cite{GWOSC}. The data we use is the publicly accessible O2 strain data from the Hanford and Livingston detectors, recorded while the detectors are in ``Science Mode''. We obtain the data using \gwpy~\cite{gwpy}. 

\subsection{Assigning \pastro to candidate events}
After the calculating the \bcr for the entire set of high-mass background and simulated triggers, we calculate probability distributions $p_b(\bcr)$ and $p_s(\bcr)$ for each 2-week time-frame of O2 data. These distributions are used to `tune' prior-odd $\alpha$ and $\beta$ values.

Using the tuned prior odds the \bcr for the candidate events can be calculated. Fig.~\ref{fig:bcrCdf} shows the \bcr distributions for the background triggers, simulated triggers and candidate events. The bulk of the background and simulated trigger distributions are separate but slightly overlap due to some of the simulated signal's being very faint. The separation suggests that the \bcr can successfully distinguish signals from noise or glitches. The vertical lines in Fig.~\ref{fig:bcrCdf} displays the \bcr for gravitational-wave candidate events. On comparing the candidate event \bcr values with the background distribution, we can estimate \pastro values for the candidate events. 

\begin{figure}[!ht]
{\centering \includegraphics[width=0.85\linewidth]{images/reweighted_bcr_cdf_smaller_legend.png} }
\caption[BCR distribution example]{Histograms represent the survival function (1-CDF) from our high-mass selection of background triggers (grey) and simulated signals (blue) triggers obtained from \pycbc's search of data from $\text{August 13 - 21, 2017}$. Vertical lines mark the $\text{ln}\ \bcr$ of IAS's GW170817A and GWTC-1's GW170814.}\label{fig:bcrCdf}
\end{figure}


\section{\label{sec:Results}Results}

\begin{table*}
\centering
\caption[Table summarising various pipeline \pastro values]{
    Table summarising the \pastro of events and candidate events from various detection pipelines, 
    along with the PyCBC triggers with $\pastrobcr > 0.5$. Note that only the 
    triggers and events within our prior space are displayed. The various pipeline \pastro represented in this table, 
    \pastroext, are from the following pipelines: 
    PyCBC ${}^{\ddagger}$ \cite{GWTC1}, 
    GstLAL ${}^{\|}$ \cite{GWTC1}, 
    PyCBC OGC-2${}^{\dagger}$ \cite{pycbc_ogc_2} 
    and IAS ${}^{\ast}$ \cite{IAS1, IAS2}. 
    Note that the catalogues labelled IAS-1 and IAS-2 correspond to the candidates published in \citet{IAS1} and \citet{IAS2}.
    \av{I am using the same symbols as Pratten et al. Should i state that I am following their labelling convention? i need to add Pratten et al vals}
    }
\label{tab:results}
\def\arraystretch{1.5} 
 \setlength{\tabcolsep}{0.5em}
\begin{tabular}{ll|c|c|c}

    Event &     Catalogue & \pastrobcr &                \pastroext &           GPS \\
\hline
 GW170104 &        GWTC-1 &        0.93 &  $1.00^{|};1.00^{\ddagger}$ & 1167559934.60 \\
 GW170121 &         IAS-1 &        0.72 & $1.00^{\dagger};1.00^{\ast}$ & 1169069152.57 \\
 GW170202 &         IAS-1 &        0.46 & $0.13^{\dagger};0.68^{\ast}$ & 1170079033.71 \\
GWC170222 & - &        0.53 &                          - & 1171814474.97 \\
GWC170302 &         IAS-1 &        0.73 &               $0.45^{\ast}$ & 1172487815.48 \\
 GW170304 &         IAS-1 &        0.83 & $0.70^{\dagger};0.99^{\ast}$ & 1172680689.36 \\
 GW170402 &         IAS-2 &        0.22 &               $0.68^{\ast}$ & 1175205126.57 \\
 GW170403 &         IAS-1 &        0.43 & $0.03^{\dagger};0.56^{\ast}$ & 1175295987.22 \\
GWC170412 &        GWTC-1 &        0.82 &               $0.06{}^{|}$ & 1176047817.00 \\
 GW170425 &         IAS-1 &        0.31 & $0.21^{\dagger};0.77^{\ast}$ & 1177134830.19 \\
 GW170608 &        GWTC-1 &        0.95 &  $0.92^{|};1.00^{\ddagger}$ & 1180922492.50 \\
 GW170727 &         IAS-1 &        0.93 & $0.99^{\dagger};0.98^{\ast}$ & 1185152686.02 \\
 GW170729 &        GWTC-1 &        0.94 &  $0.98^{|};0.52^{\ddagger}$ & 1185389805.33 \\
 GW170809 &        GWTC-1 &        0.99 &  $0.99^{|};1.00^{\ddagger}$ & 1186302517.75 \\
 GW170814 &        GWTC-1 &        1.00 &  $1.00^{|};1.00^{\ddagger}$ & 1186741859.53 \\
GW170817A &         IAS-2 &        0.76 &               $0.86^{\ast}$ & 1186974182.72 \\

\end{tabular}
\end{table*}

We analyse the $60,996$ background, $5,146$ simulated, and $25$ candidate triggers reported by \pycbc's search on the data from LIGO's second observing run, restricting our analysis to the triggers that fall within our mass-space as described in Section~\ref{sec:method}.We also analyse events and candidate events reported by GWTC-1 and the IAS group (note: some of these were identified as candidates by the \pycbc search). In Table~\ref{tab:results}, we summarise the \pastrobcr, along with the \pastro of other pipelines for comparison. Although the various pipeline \pastro are not mathematically equivalent, by comparing pipeline \pastro values for a given candidate, we can compare how significant each pipeline deems various candidates. The $\alpha$ and $\beta$ values utilised for each time-frame are reported in Appendix~\ref{apdx:alphabeta}.

\subsection{GWTC-1 Events}
All the confirmed gravitational-wave events from binary black hole mergers reported in GWTC-1 and within our prior space, (specifically GW170104, GW170608, GW170729, GW170809 and GW170814), have $\pastrobcr$ greater than $0.9$, indicating a high probability of the presence of an astrophysical signal. 

In addition to the above confirmed gravitational-wave events from GWTC-1, we have also analysed several candidate events from GWTC-1, most of which have low \pastrobcr. For example, consider the candidate event 170412, assigned a $\pastro$ of $0.06$ by \gstlal and has a $\pastrobcr$ of $0.01$. This candidate was reported to be excess power caused due noise appearing non-stationary between 60-200 Hz~\cite{GWTC1}. This candidate acts as an example of how $\pastrobcr$ may be utilised to eliminate candidates originating from terrestrial noise sources.

\subsection{IAS Events}
Our analysis of the high-mass IAS events and candidates in O2 has resulted in three events with disfavourable $\pastrobcr<0.5$ (GWC170402, GW170403, GW170425), and four events and one candidate with $\pastrobcr\geq0.5$ (GW170121, 170302, GW170304, GW170727, GW170817A). \rs{suggest an additional sentence comparing the numbers with greater/less than 0.5 to the numbers that IAS find.}

GWC170402, detected by \citet{IAS2}, is reported to have a signal that is not described well by waveforms for circular binaries with aligned spins \rs{what does "not well described" mean?}. Hence, we might have received a low \pastrobcr due to our usage of \imrphenomp, a waveform that does not account for eccentricity. Additionally, the search conducted by \citet{IAS2} was a single-detector search. Our ranking statistic relies on the signal to appear coherent, even if just faintly coherent, amongst the various detectors to have a high \pastrobcr. The lack of coherence and the non-eccentric waveform may be the leading factors for a low \pastro. GW170403 and GW170425 which have $\pastrobcr<0.35$ also have low  \pastro reported by \citet{pycbc_ogc_2},  suggesting that these events may have been false alarms.

From the candidates with $\pastrobcr>0.5$, GW170727 and 170302 are of particular interest, with \pastrobcr of $0.92$ and $0.63$. GW170727 was emitted from a black hole binary system with a source frame total mass $\approx 70\ \msun$. In addition to the high \pastrobcr reported by our study, \citet{IAS1} and \citet{pycbc_ogc_2} have also reported high \pastro values of 0.98 and 0.99, making it a viable gravitational-wave event candidate. Similarly, the sub-marginal-candidate 170302 reported by \cite{IAS1} with a \pastro of 0.45 appears to have a higher significance from our analysis, resulting in a \pastrobcr of $0.63$.  

% \av{Interesting that GW170817A has high \pastrobcr as IAS2's pipeline works well on single detector events (marginal in other detectors)} 


\subsection{New Candidate Events}
Although no clear detections are made with the \bcr, a marginal-candidate 170222 has been discovered with a $\pastrobcr\sim0.5$. This candidate has its similar masses when compared to those of GWTC-1. The remaining coherent trigger candidates all have $\pastrobcr\ll0.5$ making them unlikely to originate from astrophysical sources. 

\avi{mass and spin estimates, is it a vanilla bbh}


\section{\label{sec:Conclusion}Conclusion}


% Gravitational waves from intermediate-mass black holes will Current ground-based gravitational-wave observatories like LIGO can probe the lower-end of the intermediate-mass range, u, while future space based detectors will be able to study the mergers of large limits on the low end of this mass regime are continuing to improve. LIGO found 190521, can find more if we are more sensitive, and can better harness data. 


% Most of what we know about intermediate mass black holes so far has been informed from electromagnetic observations. Gravitational waves provide a complementary lens to study these massive objects. In the future, LISA will fill in this gap, but we can begin now with LIGO... In this paper, we 


In this paper, we demonstrate that the Bayesian Coherence Ratio~\cite{BCR1} can be used as a ranking statistic to provide a better measure of significance for gravitational-wave candidates by re-analysing the significance of high-mass binary black hole triggers from O2. This method takes a step towards building a unified Bayesian framework that provides a search-pipeline agnostic measure of significance, utilising the same level of physical information incorporated during parameter estimation. 

We focused our analysis on the high-mass regime as this region of the parameter space is plagued with a high number of short duration terrestrial artefacts that can mimic signals. In addition to the high-mass triggers, we also analyse the high-mass binary black hole events in O2 reported by LIGO~\cite{GWTC1} and IAS~\cite{IAS1, IAS2}. Using \pastrobcr, we find that the analysed GWTC-1 events have high probabilities of originating from an astrophysical source. We also find that some of the GWTC-1 marginal triggers that have corroborated terrestrial sources (for example candidate 170412) have low \pastrobcr, indicating this method's ability to discriminate between terrestrial artefacts and astrophysical signals. Our analysis on the IAS events has demonstrated that GW17072 is highly likely to originate from an astrophysical source, while GW17040 is not. Finally, we did not identify any new gravitational-wave events, but we did find some a marginal candidate 170222. 

Although our analysis targets high-mass triggers, this method can be extended to include the entire body range of LIGO-detectable gravitational-wave sources. Additionally, to further improve the method's infrastructure, we can use more robust gravitational-wave templates (such as templates that incorporate higher-order modes), and sophisticated glitch models. Future analysis can also incorporate data from all available detectors in a network to increase the sensitivity of \pastrobcr. The BCR can discern better whether a candidate is a coherent astrophysical candidate or an incoherent glitch with data from more detectors. 

\avi{as the core of this method is PE, improvements in PE can be adapted into this method}

%%%%%%---SECTIONS-END---%%%%%%%%%%%%

%%%%%%---ACKNOWLEDGEMENTS---%%%%%%%%%%%%
\begin{acknowledgments}

This research has made use of data, software and/or web tools obtained from the Gravitational Wave Open Science Center (https://www.gw-openscience.org), a service of LIGO Laboratory, the LIGO Scientific Collaboration and the Virgo Collaboration. LIGO is funded by the U.S. National Science Foundation. Virgo is funded by the French Centre National de Recherche Scientifique (CNRS), the Italian Istituto Nazionale della Fisica Nucleare (INFN) and the Dutch Nikhef, with contributions by Polish and Hungarian institutes.

Thank Stuart Anderson, CIT

 We greatly appreciate the contributions of all these computing allocations. All pe performed for this study including test runs and failed simulations used about 2.5M core-hours which under the assumption of 30W per core-hour and a CO2 intensity of electricity of ~ 600 kg CO2 per MWh amounts to a carbon footprint of~ 45t of CO2. 



\end{acknowledgments}
%%%%%%%%%%%%%%%%%%%%%%%%

\appendix



\section{Bayesian Evidence Evaluation}\label{sec:bayesianEvidEval}
\subsection{Noise Model}
We assume that each detector's noise is Gaussian and stationary over the period being analysed~\cite{ligo_psd}. In practice, we assume that the noise has a mean of zero that the noise variance $\sigma^2$ is proportional to the noise power spectral density (PSD) \psd of the data. Using the \psd, for each data segment $d_i$ in each of the $i$ detectors in a network of $D$ detectors, we can write 
\begin{equation}
\label{eq:zn}
Z^N_i = \mathcal{N}(d_i) = \frac{1}{2\pi \psd_i} \ \text{exp}\left(-\frac{1}{2} \frac{d_i}{\psd_i} \right) \ ,
\end{equation}
where $\mathcal{N}(d_i)$ is a normal distribution with $\mu=0$ and $\sigma^2\sim \psd$. 

\subsection{Coherent Signal Model}
We model coherent signal using a binary black hole waveform template \template, where the vector \parameters contains a point in the 15 dimensional space describing precessing binary-black hole mergers. For the signal to be coherent, \parameters must be consistent in each 4-second data segment $d_i$ for a network of $D$ detectors, Hence, the coherent signal evidence is calculated as
\begin{equation}
\label{eq:zs}
Z^S = \int\limits_{\parameters} \prod\limits^{D}_{i=1} \left[ \mathcal{L}(d_i|\template)\right] \pi(\parameters | \mathcal{H}_S)\  \text{d}\parameters \ ,
\end{equation}
where $\pi(\parameters| \mathcal{H}_S)$ is the prior for the parameters in the coherent signal hypothesis, and $\mathcal{L}(d_i|\template))$ is the likelihood for the coherent signal hypothesis that depends on the gravitational-wave template \template and its parameters \parameters. 

\subsection{Incoherent Glitch Model}
Finally, as glitches are challenging to model and poorly understood, we follow \citet{bci} and utilise a surrogate model for glitches: the glitches are modelled using gravitational-wave templates  \template with uncorrelated  parameters amongst the different detectors such that  $\parameters_i \neq \parameters_j$ for two detectors $i$ and $j$~\cite{bci}.  Modelling glitches with \template captures the worst case scenario: when glitches are identical to gravitational-wave signals (excluding coherent signals). Thus, we can write $Z^G_i$ as 
\begin{equation}
\label{eq:zg}
Z^G_i = \int\limits_{\parameters} \mathcal{L}(d_i|\template)\ \pi(\parameters| \mathcal{H}_G)\  \text{d}\parameters  \ ,
\end{equation}
where $\pi(\theta| \mathcal{H}_G)$ is the prior for the parameters in the incoherent glitch hypothesis. 



\section{Tuning the prior-odds}\label{sec:tuning-prior-odds}

After calculating the \bcr for a set of background triggers and simulated triggers from as stretch of detector-data (a data chunk), we can compute probability distributions for the background and simulated triggers, $p_b(\bcr)$ and $p_s(\bcr)$. We expect the background trigger and simulated signal \bcr values to favor the incoherent glitch and the coherent signal hypothesis, respectively. Ideally, these distributions representing two unique populations should be distinctly separate and have no overlap in their \bcr values. The prior odds parameters $\alpha$ and $\beta$ from Eq.~\ref{eq:bcr} help separate the two distributions. Altering $\alpha$ translates the \bcr probability distributions while adjusting $\beta$ spreads the distributions. Although Bayesian hyper-parameter estimation can determine the optimal values for $\alpha$ and $\beta$, an easier approach is to adjust the parameters for each data chunk's \bcr distribution. In this study, we tune $\alpha$ and $\beta$ to maximally separate the \bcr distributions for the background and simulated triggers. 

To calculate the separation between $p_b(\bcr)$ and $p_s(\bcr)$, we use the Kullback--Leibler divergence (KL divergence) $D_{KL}$, given by
\begin{equation}
    D_{KL}(p_b | p_s) = \sum\limits_{x\in \bcr} p_b(x) \log \left( \frac{p_b(x)}{p_a(x)} \right)  \ .
\end{equation}
The $D_{KL}=0$ when the distributions are identical and increases as the asymmetry between the distributions increases. 

We limit our search for the maximum KL-divergence in the $\alpha$ and $\beta$ ranges of $[10^{-10}, 10^0]$ as values outside this range are nonphysical. We set our values for $\alpha$ and $\beta$ to those which provide the highest KL-divergence and calculate the \bcr for candidate events present in this data chunk. Note that we conduct the analysis in data chunks of a few days rather than an entire data set of a few months as the background may be different at different points of the entire data set.

\section{Marginalizing over PSD statistical uncertainties}\label{sec:psd-marginalization}
To generate the results in Fig.~\ref{fig:bcrCdf}, we applied a post-processing step to marginalize the uncertainty in the PSD. In Fig.~\ref{fig:bcrCdfUnmarginalized}, we show the results if this post-processing step is not applied. Clearly, marginalizing over uncertainty in the PSD yields an improvement in the separation of the noise and signal distributions. Quantitatively, at a threshold $\bcr_T=0$ the post-processing step results in a reduction in the number of background $\bcr > \bcr_T$ from $60.7\%$ to $25.28\%$ in the August 13 - 21, 2017 time-frame of data. For the entirety of O2 PSD marginalization resulted in a $49.26\%$ improvement in search efficiency. 

\begin{figure}[!ht]
{\centering \includegraphics[width=0.85\linewidth]{images/orig_bcr_cdf_smaller_legend.png} }
\caption[BCR distribution example]{This plot is analogous to Fig.~\ref{fig:bcrCdf}, but without using the post-processing step to marginalize over PSD statistical uncertainties. Without the post-processing step, there is a greater overlap between the background (grey) and foreground (blue) survival functions. For more details about this plot, refer to the caption of Fig.~\ref{fig:bcrCdf}.}\label{fig:bcrCdfUnmarginalized}
\end{figure}




\section{Tuned prior odds}\label{apdx:alphabeta}

O2 lasted several months over which the detector's sensitivity varied. Hence, a part of our analysis entailed tuning the prior odds for obtaining a signal and a glitch, $\alpha$ and $\beta$, as described in Section~\ref{sec:method}. Table~\ref{tab:priorodds} presents the signal and glitch prior odds utilised for each time-frame of O2 data. 
\begin{table}
\centering
\caption{The prior odds used for each time-frame of data from O2. Each time frame commences at the start date and concludes at the following time-frame's start date.
    }
\label{tab:priorodds}
\def\arraystretch{1.5} 
\setlength{\tabcolsep}{0.5em}
\begin{tabular}{c|cc}

 Start Date &    $\hat{\pi}^S$ &    $\hat{\pi}^G$ \\
\midrule
 2016-11-15 &        - &        - \\
 2016-11-30 &        - &        - \\
 2016-12-23 & 1.00E+00 & 6.25E-01 \\
 2017-01-22 & 1.00E+00 & 2.33E-02 \\
 2017-02-03 & 1.00E-10 & 2.44E-01 \\
 2017-02-12 & 1.76E-08 & 5.96E-02 \\
 2017-02-20 & 6.55E-10 & 2.22E-03 \\
 2017-02-28 & 1.00E-10 & 5.96E-02 \\
 2017-03-10 & 2.56E-10 & 3.91E-01 \\
 2017-03-18 & 1.60E-10 & 1.00E+00 \\
 2017-03-27 & 1.10E-08 & 5.96E-02 \\
 2017-04-04 & 3.73E-02 & 2.33E-02 \\
 2017-04-14 & 1.05E-09 & 2.44E-01 \\
 2017-04-23 & 2.68E-09 & 1.46E-02 \\
 2017-05-08 & 1.00E+00 & 2.44E-01 \\
 2017-06-18 & 6.55E-10 & 3.39E-04 \\
 2017-06-30 & 2.02E-05 & 5.69E-03 \\
 2017-07-15 & 1.05E-09 & 9.54E-02 \\
 2017-07-27 & 1.00E+00 & 2.12E-04 \\
 2017-08-05 & 2.12E-04 & 3.73E-02 \\
 2017-08-13 & 2.68E-09 & 8.69E-04 \\
 2017-08-21 &        - &        - \\

\end{tabular}
\end{table}


Tuning the prior odds can dramatically affect the \pastrobcr. For example, consider Table~\ref{tab:tuningresults}, which reports tuned \pastrobcr and un-tuned \untunedpastrobcr (where $\alpha=1$ and $\beta=1$) for various high-mass events and candidates. By tuning the prior odds, the \pastrobcr for some IAS events (for example, GW170403 and GW170817A) can change by more than 0.5, resulting in the promotion/demotion of a candidate's significance.

\begin{table}
\centering
\caption{Table of \pastrobcr using ``tuned'' prior odds and  \pastrobcr using uninformed prior odds of 
$\hat{\pi}^S=1$ and $\hat{\pi}^G=1$ (represented by \untunedpastrobcr).  
Details of other columns provided in Table~\ref{tab:results}.}
\label{tab:tuningresults}
\def\arraystretch{1.5} 
 \setlength{\tabcolsep}{0.5em}
\begin{tabular}{ll|c c| c}

     Event & Catalog & \tunedpastrobcr & \untunedpastrobcr &          \tc \\
\hline
  GW170104 &  GWTC-1 &             0.97 &               0.95 & 1167559936.60 \\
  GW170121 &   IAS-1 &             0.83 &               0.68 & 1169069154.57 \\
    170209 &       - &             0.32 &               0.00 & 1170659643.47 \\
    170222 &       - &             0.58 &               0.50 & 1171814476.97 \\
    170302 &   IAS-1 &             0.78 &               0.54 & 1172487817.48 \\
  GW170304 &   IAS-1 &             0.94 &               0.80 & 1172680691.36 \\
 GWC170402 &   IAS-2 &             0.60 &               0.00 & 1175205128.57 \\
  GW170403 &   IAS-1 &             0.54 &               0.90 & 1175295989.22 \\
    170421 &       - &             0.27 &               0.21 & 1176789158.14 \\
  GW170425 &   IAS-1 &             0.22 &               0.16 & 1177134832.18 \\
  GW170608 &  GWTC-1 &             0.99 &               0.99 & 1180922494.50 \\
  GW170727 &   IAS-1 &             0.98 &               0.99 & 1185152688.02 \\
  GW170729 &  GWTC-1 &             0.98 &               0.95 & 1185389807.30 \\
  GW170809 &  GWTC-1 &             0.99 &               0.99 & 1186302519.75 \\
  GW170814 &  GWTC-1 &             1.00 &               1.00 & 1186741861.53 \\
 GW170817A &   IAS-2 &             0.92 &               0.30 & 1186974184.72 \\

\end{tabular}
\end{table}


\bibliography{high_mass_bib}% Produces the bibliography via BibTeX.

\end{document}
%
% ****** End of file apssamp.tex ******
